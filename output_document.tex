\documentclass[11pt]{article}

\usepackage[a4paper, total={6in, 8in}]{geometry}
\usepackage[english,czech]{babel}
\usepackage{listings}
\usepackage{xcolor}

\author{Tomáš Diblík, Vilém Cerman, Adam Pečenka \\ 3.A}
\title{Jarníkův algoritmus a Borůvkův algoritmus}
\date{\selectlanguage{czech}\today}

\definecolor{codegreen}{rgb}{0,0.6,0}
\definecolor{codegray}{rgb}{0.5,0.5,0.5}
\definecolor{codepurple}{rgb}{0.58,0,0.82}
\definecolor{backcolour}{rgb}{0.95,0.95,0.92}
\lstdefinestyle{code_style}{
    backgroundcolor=\color{backcolour},   
    commentstyle=\color{codegreen},
    keywordstyle=\color{magenta},
    numberstyle=\tiny\color{codegray},
    stringstyle=\color{codepurple},
    basicstyle=\ttfamily\footnotesize,
    breakatwhitespace=false,         
    breaklines=true,                 
    captionpos=b,                    
    keepspaces=true,                 
    numbers=left,                    
    numbersep=5pt,                  
    showspaces=false,                
    showstringspaces=false,
    showtabs=false,                  
    tabsize=2
}
\lstset{style=code_style}

\begin{document}

\maketitle
\thispagestyle{empty}
\pagebreak

% TODO: 
% - Seznámení Jar (teorie, neobsahovat implementaci)
% - Seznámení Bor (teorie, neobsahovat implementaci)
% - Historie Jar (jak byl vymýšlenej a atd
% - Historie Bor (jak byl vymýšlenej a atd
% - Využití (obecný, jelikož řeší stejnej problém)
% - Rozdíl mezi ("Jarníkův a Borůvkův algoritmus" -- Primův) A (Jarníkův algortimus ; Borůvkův algoritmus)
% - Implementace (pro každej tyhle body)
%   - v pseudo kódu ---- vysvětlení
%   - v C# (aby všichni hned nestratili pozornost)
%   - v C

\tableofcontents
\thispagestyle{empty}
\pagebreak

\setcounter{page}{1}
\pagenumbering{Roman}
\section{Sdílené funkce}
Abychom zbytečně neopakovali nějaké kusy kódu, tak
zde vypíši funkce a globální promněné, které většina sortů používá.
Tyto funkce poté již nebudu v implementaci uvádět.
Toto rozhodnotí jsem učinil z důvodu délky práce.

% Example code
\subsection{Vždy využité importy}
\begin{lstlisting}[language=C]
#include <stdio.h>
#include <stdlib.h>
\end{lstlisting}

% Example code
\subsection{Globální definice (konstanty)}
\begin{lstlisting}[language=C]
#define NUMBER_OF_VALUES 20
#define RANDOM_RANGE_MIN 1
#define RANDOM_RANGE_MAX 40
\end{lstlisting}

% Example code
\subsection{Funkce print\_array}
\begin{lstlisting}[language=C]
// Function signature
void print_array(int *values, int arr_size);

// Function implementation
void print_array(int *values, int arr_size) {
  for (int i = 0; i < arr_size; i++) {
    printf("%d", values[i]);
    if (i != arr_size - 1) {
      printf(", ");
    }
  }
  printf("\n");
}
\end{lstlisting}

\subsection{Funkce get\_random\_values}
\begin{lstlisting}[language=C]
// Function signature
int* get_random_values();

// Function implementation
int* get_random_values() {
  srand(time(NULL));
  int *random_values = (int *)malloc(sizeof(int) * NUMBER_OF_VALUES);
  for (int i = 0; i < NUMBER_OF_VALUES; i++) {
    random_values[i] =
        rand() % ((RANDOM_RANGE_MAX + 1) - RANDOM_RANGE_MIN) + RANDOM_RANGE_MIN;
  }
  return random_values;
}
\end{lstlisting}

\pagebreak

% Example section and subsection
\section{Selection sort}
\subsection{Základní informace}
Selection sort je jeden z nejjednoduších a nejznámější třídících algoritmů.
Dnes již skoro nepoužíván, vzhledem k existenci optimálnější algoritmů,
avšak pořád učen, vzhledem k své jednoduchosti.
Historie selection sortu není dobře zdokumentována, tudíž je původ algoritmu nejasný.
Selection sort funguje tak, že jdeme prvek po prvku, aktuální prvek poté porovnáme
s zbytkem prvků a zjistíme která hodnota na jakém indexu je nejmenší.
Tuto nejmenší hodnotu poté prohodíme s aktuálním prvkem.

% Example algorithm complexity
\subsection{Komplexity}
\medbreak
\textbf{Nejhorší možná časová komplexita}: $O(n^2)$
\medbreak\noindent
\textbf{Průměrná časová komplexita}: $O(n^2)$
\medbreak\noindent
\textbf{Nejlepší možná časová komplexita}: $O(n^2)$
\medbreak\noindent
\textbf{Prostorová komplexita}: $O(1)$
\medbreak\noindent
\textbf{Stabilita}: Nestabilní
\medbreak

% Example section
\subsection{Implementace}
Jedná se o (skoro) typickou implementaci selection sortu, avšak tato implementace 
obsahuje 2 optimalizace. První je, že implementace neprochází poslední element, 
druhou je, že pokud se nejmenší index rovná aktuálnímu indexu tak 
neprovádí prohození hodnot.

% Example code
\begin{lstlisting}[language=C]
void selection_sort(int *values, int arr_size);

int main() {
  int *random_values = get_random_values();

  print_array(random_values, NUMBER_OF_VALUES);
  selection_sort(random_values, NUMBER_OF_VALUES);
  print_array(random_values, NUMBER_OF_VALUES);

  free(random_values);
  return 0;
}

void selection_sort(int *values, int arr_size) {
  int smallest_index;
  for (int i = 0; i < arr_size - 1; i++) {

    smallest_index = i;

    // if the current element is smaller than the current smallest, update smallest_index
    for (int j = i + 1; j < arr_size; j++) {
      if (values[j] < values[smallest_index]) {
        smallest_index = j;
      }
    }

    // if smallest_index has changed, swap the elements
    if (smallest_index != i) {
      int temp = values[smallest_index];
      values[smallest_index] = values[i];
      values[i] = temp;
    }
  }
}
\end{lstlisting}

\end{document}